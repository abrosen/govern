\documentclass[11pt,letterpaper]{article}
\usepackage[latin1]{inputenc}
\usepackage{amsmath}
\usepackage{amsfonts}
\usepackage{amssymb}
\usepackage{graphicx}
\usepackage[left=1.00in, right=1.00in, top=1.00in, bottom=1.00in]{geometry}
\author{Andrew Rosen, Brendan Benshoof, Anu G. Bourgeois, Robert W. Harrison}
\date{}
\title{How to Build Distributed Governed Systems}
\begin{document}
	\maketitle
	
	
	Developers and researchers typically use one of two models when building a distributed system.
	The first is a completely centralized model, where one server or person is in control.
	The other alternative is a completely decentralized model, where no single entity has control, except for perhaps the protocol itself.
	
	In practice, this is not the only models that exist in the wild.  
	
	
	How can we create a system that balances multiple powerful entities against each other.
	
	
	
	Nomic is a game created by Peter Suber in 1982 \cite{nomic}.
	The initial rules of the game define a goal (gaining points) and describe how Nomic works.
	Each ``move'' in the game is a proposition to add, remove, or amend the rules of the game (unless of course, the rules have defined new actions that players can take for their turns).
	Any modification to the rules is then voted upon by the players and goes into effect (again, unless the rules have moved the system away from a democracy).
	The player then collects points.
	
	Previous work has been done on using Nomic as the system for blockchain based systems for Bitcoin or other cryptocurrencies \cite{tezos} using a constructed language.
	We want to show how Nomic can be used as the basis for building a wide variety self-governing distributed systems from blockchains, to administrating servers, to solving very interesting problems in managing DNS and security, such as key revocation.
	Governomic also does not assume the need for a constructed language.
	
	Because Nomic is a system for laying down rules, defining who can do what and who has power, we can use it to simulate just about any form of governmental system implemented by humans. 
	Some examples:
	\begin{itemize}
		\item An autocracy, a system where power and authority is vested in a single entity. 
		\item An aristocracy, where only chosen members of the system get to vote, or even have a voice.
		\item A plutocracy, where a large sum of currency is needed to either participate or take an action. In other words, he who has the gold makes the rules.
		\item A representative democracy, where users in the system collectively appoint individuals to act as their proxy or proxies.
		\item A pure democracy, where each users can or must vote on each issue.
	\end{itemize}
	
	There are many variations that can be taken from history.  
	For example, a representative democracy might elect to temporarily appoint a dictator with a wide range of powers, but limited in certain capacities, much like the Roman dictator and general Lucius Quinctius Cincinnatus.
	There are number of methods that would be impractical  in a non-computing  environment, but can be managed here, such as choosing a new dictator every 5 minutes.
	
	
	\section{Creating Your Own Nomic Based System}
	
	
	Nomic's initial rules are extremely simple, but allow us to create an incredibly complex systems.
	Likewise, we will present a abstract way to build a Nomic based system.
	
	Implementing a Nomic system can be thought of building 4 different components.
	\begin{itemize}
		\item The \textit{state} of system.  This is the current rules as well as any information needed about participants.
		\item The \textit{Validator}  is  a function which takes in the proposed new block and  returns true or false, depending on whether on not it meets the criteria accepted and executed.
		\item The \textit{Executor} is a function which makes the proposed changes to the the rules and the state of the system.  It also performs any action that needs to be taken independent of the success or failure of a rule change or some other action.
		\item A networking component to communicate with the other members of the system.
	\end{itemize}
	
	The first thing is to determine your initial ruleset.
	We recommend that the initial ruleset be built from the ground up using a pure democracy, much like the base rules in Nomic.
	This allows the specific rules for this particular system to be built from the ground up.
	The specific starting rules depend on each use case.
	
	
	\section{Building a Decentralized Identity Management System}
	
	Begin by creating a system of basic rules for management.
	 
	
	
	
	\section{Building a Blockchain}
	
%	\section{Building a New DNS Service}
	
%	One great advantage over this system is there is modifiable record of whose keys are valid.
	
%	Should a root server become compromised or start issuing invalid keys, the other servers can act and revoke the signer's key, rendering their signed records invalid.
	
	
%	In this scenario, we have two sets of identities stored by Governomic: the DNS records themselves, and those who manage the DNS records.
%	Each root DNS server can autonomously add any DNS record and sign it by with their key.
	
%	What makes the system different from regular DNS is the ability to autorize new 
	
	\section{Auditable And Highly Controlled System}
	
	
	
	
	\section{Addressing Security Concerns}
	foo \cite{camilleri2010playing}
	 \cite{tezos}
	\bibliography{refs}
	\bibliographystyle{IEEEtran}
\end{document}